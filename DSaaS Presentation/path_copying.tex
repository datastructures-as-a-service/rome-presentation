\begin{frame}{path-copying}
	\centering
	Achieve full persistence using a technique called \emph{path-copying}
	% ~\cite{driscoll1986making}

	\begin{figure}
		\centering
		\begin{tikzpicture}
		
		\tikzstyle{thickarrow}=[line width=1mm,-triangle 45,postaction={draw, line width=3mm, shorten >=4mm, -}]
		\tikzstyle{hl}=[color=accentcolor]
		\tikzstyle{nn}=[circle, fill, style={fill=accentcolor,text=white}]
		\tikzstyle{ns}=[nn, style={fill=grey}]

		\begin{scope}[>=stealth']

		  
		\node (a2) at (2.5,-1) [circle,draw]{$b$};
		\node (b2) at (2.5,-2) [circle,draw]{$d$};
		\node (d2) at (1.5, -2) [circle,draw]{$a$};
		\node<1,6->(r2) at (3,0) [circle,draw]{$r$};
		\node<1-2,5->(c2) at (3.5,-1) [circle,draw]{$c$};
		\node(e2) at (3.5, -2) [circle,draw]{$e$};

		\node<2-5>(r2) at (3,0) [circle,draw, hl]{$r$};
		\node<3-4>(c2) at (3.5,-1) [circle,draw, hl]{$c$};

		\draw[->] (r2) -- (a2);
		\draw[->] (a2) -- (b2);
		\draw[->] (a2) -- (d2);
		\draw[->] (c2) -- (e2);
		
		\draw<1-2, 6->[->] (r2) -- (c2);

		\draw<3-5>[->, hl] (r2) -- (c2);

		\visible<4->{\node(f) at (4.5, -2) [nn]{$f$};}
		\draw<4>[->, hl] (c2) -- (f);

		\visible<5->{\node(c3) at (4.5, -1) [nn]{$c'$};}
		\draw<5->[->, hl] (c3) -- (f);
		\draw<5->[->, hl] (c3) -- (e2);
		\draw<5>[->, hl] (r2) -- (c3);

		\visible<6->{\node(r3) at (4, 0) [nn]{$r'$};}
		\draw<6->[->, hl] (r3) -- (c3);
		\draw<6->[->, hl] (r3) -- (a2);

		\end{scope}
		\end{tikzpicture}
		\end{figure}
\end{frame}