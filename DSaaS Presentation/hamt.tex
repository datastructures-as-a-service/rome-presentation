\begin{frame}[fragile]
	\frametitle{Hashed Array Mapped Trie (HAMT)}

	\newcommand{\hl}[2]{\color<#1>{accentcolor}#2\color{primary}}

	% \begin{minipage}{0.5\textwidth}
	% 	Insert keys: \texttt{A}, \texttt{B}, \texttt{C} and \texttt{D}.

	% 	Hash \texttt{A} = \ldots\ 00000 00000 \hl{2}{00010}

	% 	\onslide<3->{Hash \texttt{B} = \ldots\ 00000 \hl{7-9}{00001}\ \hl{3-6}{00001}}

	% 	\onslide<6->{Hash \texttt{C} = \ldots\ \hl{10}{00011}\ \hl{7-9}{00011}\ \hl{6}{00001}}

	% 	\onslide<10->{Hash \texttt{D} = \ldots\ \hl{10}{00000}\ 00011 00001}

	% \end{minipage}%
	% \begin{minipage}{0.5\textwidth}
	% 	% \begin{figure}
	% 	  \centering
	% 	    \begin{tikzpicture}

	% 	    \tikzstyle{obj}=[draw, rectangle, minimum height=0.5cm, minimum width=1.5cm];
	% 	    \tikzstyle{arr}=[draw, rectangle, minimum height=0.5cm, minimum width=0.5cm];
	% 	    \tikzstyle{bigobj}=[draw, rectangle, minimum height=2.5cm, minimum width=3.5cm];
	% 	    \tikzstyle{db}=[cylinder, shape border rotate=90, draw,minimum height=2cm,minimum width=2.6cm];
	% 	    \tikzstyle{bmp}=[arr, style={fill=gray,text=white}];

	% 	    \begin{scope}[>=stealth]
	% 			\node (level0) at (0,0) {};

	% 			\node<1>(level0_2) [left=0cm of level0, arr] {Array Of Pointers};
	% 			% \node (level0_4) [right=0cm of level0_2, arr] {};
	% 			\node<1>(bitmap0) [left=0cm of level0_2, bmp] {Bitmap};
	% 			% \node (ext_1) [right=0cm of level0_4]{\ldots};
	% 			\onslide<1>{\node (rootLabel) [above left=0cm and -0.48cm of level0_2] {Node};}
	% 			\node<1>(pointOfBitmap)[below=0cm of level0_2]{\begin{varwidth}{6cm}
	% 			The bitmap indicates the presence, and is used to determine the position, of a pointer within the pointer array.
	% 			\end{varwidth}};

	% 			\node<1>(pointOfArray)[below=0cm of pointOfBitmap]{\begin{varwidth}{6cm}
	% 			The array of pointers can point to another node or a key.
	% 			\end{varwidth}};

	% 			\pause

	% 			\node(level0_2) [left=0cm of level0, arr] {};
	% 			\node<2>(bitmap0) [left=0cm of level0_2, bmp] {};


	% 			\node (level0_2) [left=0cm of level0, arr] {$\cdot$};

	% 			\node<2>(bitmap0) [left=0cm of level0_2, bmp] {\ldots 0100};
	% 			\node<2-4>(package1) [below=0.5cm of level0_2]{A};
	% 			\draw<2-4>[->] (level0_2) -- (package1);

	% 			\onslide<2>{
	% 				\node (indicator) [above = 0.5cm of bitmap0] {
	% 					\begin{varwidth}{4cm}
	% 						\small 2nd bit set, indicating that there exists a pointer in the array with a hash suffix of 2.
	% 					\end{varwidth}};
	% 				\node (anchor) [above right = -0.2cm and -0.7cm of bitmap0]{};
	% 				\draw[->] (indicator) -- (anchor);
	% 			}

	% 			\pause

	% 			\node<3-4>(a) [right = 0cm of level0_2, arr]{$?$};
	% 			\node<3-4>(b) [left = 0cm of level0_2, arr]{$?$};
	% 			\node<3>(bitmap0) [left=0cm of b, bmp] {\ldots 0100};

	% 			\pause

	% 			\node<4>(bitmap0) [left=0cm of b, bmp] {\ldots 0110};

	% 			\pause

	% 			\node<5->(bitmap0) [left=0cm of level0_2, bmp] {\ldots 0110};
	% 			\node (level0_4) [right=0cm of level0_2, arr] {$\cdot$};

	% 			\node (package1) [right=0.5cm of level0_4]{A};
	% 			\draw[->] (level0_4) -- (package1);

	% 			\onslide<5-6>{
	% 				\node (package2) [below=0.5cm of level0_2]{B};
	% 				\draw[->] (level0_2) -- (package2);
	% 			}

	% 			\pause
	% 			\pause

	% 			\node<7> (bitmap1)[below= 0.5cm of level0_2, bmp] {\ldots 0000};
	% 			\draw<7>[->] (level0_2) -- (bitmap1);

	% 			\pause

	% 			\node<8> (bitmap1)[below= 0.5cm of level0_2, bmp] {\ldots 0010};
	% 			\draw<8>[->] (level0_2) -- (bitmap1);
	% 			\node<8>(level1_B) [right=0cm of bitmap1, arr] {$\cdot$};
	% 			\node<8>(package2) [below=0.5cm of level1_B]{B};
	% 			\draw<8>[->] (level1_B) -- (package2);

	% 			\pause

	% 			\node (bitmap1) [below= 0.5cm of level0_2, bmp] {\ldots 1010};
	% 			\draw[->] (level0_2) -- (bitmap1);
	% 			\node(level1_B) [right=0cm of bitmap1, arr] {$\cdot$};
	% 			\node (level1_C) [right=0cm of level1_B, arr] {$\cdot$};
	% 			\node(package2) [below=0.5cm of level1_B]{B};
	% 			\draw[->] (level1_B) -- (package2);
	% 			\onslide<9>{\node (package3) [below=0.5cm of level1_C]{C};}


	% 			\draw[->] (level0_4) -- (package1);

	% 			\onslide<9>{\draw[->] (level1_C) -- (package3);}

	% 			\pause

	% 			\node (bitmap2) [below right = 0.5cm and -0.2cm of level1_C, bmp] {\ldots 1001};
	% 			\node (level2_1) [right=0cm of bitmap2, arr] {$\cdot$};
	% 			\node (level2_2) [right=0cm of level2_1, arr] {$\cdot$};

	% 			\node (package3) [below=0.5cm of level2_2]{C};
	% 			\node (package4) [below=0.5cm of level2_1]{D};

	% 			\draw[->] (level2_2) -- (package3);
	% 			\draw[->] (level1_C) -- (bitmap2);
	% 			\draw[->] (level2_1) -- (package4);

	% 	    \end{scope}
	% 	    \end{tikzpicture}
	% 	  % \caption{An example of a typical HAMT, where the grey blocks are the bitmaps (indicating the presence of pointers), the white blocks are the array block cells filled with the label of the edge leading to next node or value, and the black dots are the values that this trie store.
	% 	  % The hashed keys of this trie are 2\ldots, 42\ldots and 4E1\ldots in hexadecimal (the rest of the key identifiers are irrelevant for this example trie). }
	% 	  % \label{fig:hamt-diagram}
	% 	% \end{figure}
	% \end{minipage}

	\begin{figure}
	  \centering
	    \begin{tikzpicture}

	    \tikzstyle{obj}=[draw, rectangle, minimum height=0.5cm, minimum width=1.5cm];
	    \tikzstyle{arr}=[draw, rectangle, minimum height=0.5cm, minimum width=0.5cm];
	    \tikzstyle{bigobj}=[draw, rectangle, minimum height=2.5cm, minimum width=3.5cm];
	    \tikzstyle{db}=[cylinder, shape border rotate=90, draw,minimum height=2cm,minimum width=2.6cm];
	    \tikzstyle{bmp}=[arr, style={fill=gray,text=white}];

	    \begin{scope}[>=stealth]
	      \node (level0) at (0,0) {};
	      \node (level0_2) [left=0cm of level0, arr] {};
	      \node (level0_4) [right=0cm of level0_2, arr] {};
	      \node<1-2>(bitmap0) [left=0cm of level0_2, bmp] {\ldots000000};

	      % \node (rootLabel) [above=0.5cm of level0_2] {Root Node};
	      \node (bitmapLabel)[above=0cm of bitmap0] {Bitmap};
	      \node (refArrayLabel) [right=0cm of level0_4]{Reference Array};

	      \pause

	      \node (hashkey1) [below=2cm of bitmap0 ]{$h(k_{1}) = 00000\ldots$};

	      \pause

	      \node<3>(bitmap0) [left=0cm of level0_2, bmp] {\ldots000001};

	      \node (package1) [below left=0.2cm of level0_2]{$(k_{1}, v_{1})$};
	      \draw[->] (level0_2.south) -- (package1);

	      \pause

	      \node (bitmap0) [left=0cm of level0_2, bmp] {\ldots001001};
	      \node<4>(package2) [below right=0.2cm of level0_4]{$(k_{2}, v_{2})$};
	      \draw<4>[->] (level0_4.south) -- (package2.north);
	      \node (hashkey2) [below=0cm of hashkey1]{$h(k_{2}) = 00011$ $00011\ldots$};

	      \pause

	      \node (hashkey3) [below=0cm of hashkey2]{$h(k_{3}) = 00011$ $00101\ldots$};


	      \node (level1) at (2, -1) {};
	      \node (level1_0) [left=0cm of level1, arr] {};
	      \node (level1_5) [right=0cm of level1_0, arr] {};
	      % \node<2>(bitmap1) [left=0cm of level1_0, bmp] {\ldots000000};
	      \node (bitmap1) [left=0cm of level1_0, bmp] {\ldots101000};

	      \draw[->] (level0_4) -- (level0_4 |- bitmap1.north);

	      \node (package2) [below left=0.2cm of level1_0]{$(k_{2}, v_{2})$};
	      \node (package3) [below right=0.2cm of level1_5]{$(k_{3}, v_{3})$};

	      \draw[->] (level1_0.south) -- (package2);
	      \draw[->] (level1_5.south) -- (package3);

	    \end{scope}
	    \end{tikzpicture}
	  \caption{An example of an HAMT\@. The grey blocks represent the bitmaps, and the white cells represent the array of references to key--value pairs stored in this trie.}
	  %, and the key--value pairs stored by this trie are \todo{represented as the alphabet characters}.}
	  \label{fig:hamt-diagram-path-copying}
	\end{figure}
\end{frame}