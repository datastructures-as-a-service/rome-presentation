\begin{frame}[fragile]
	\frametitle{Adapted Flux Architecture}
	\note{The browser application uses an adapted Flux architecture, and ReactJS for rendering the HTML views.
	The Flux architecture is very straightforward, the idea is to have different units, where each unit receive information, processes the information and passes it on to the next one.
	Therefore it is a one-way data-flow of information and very easy to debug once things go wrong.}

\centering
\begin{tikzpicture}
\tikzstyle{obj}=[draw, rectangle, minimum height=1cm, minimum width=1.8cm];
\begin{scope}[>=stealth]
  \node (dispatcher) at (-3.2,-2) {Emits Data};
  \node (action) at (1,-2) [obj]{Action Dispatcher};
  \node (generate) [above=0.5cm of action]{Generates Action};
  \node (views) [above=0.5cm of generate, obj]{View};
  \node (inform) [left=0.5cm of views]{Notifies};
  \node (stores) [left=0.5cm of inform, obj]{Store};
  \node (data) [right=0.5cm of action]{Data};
  \node (webutils) [right=0.5cm of data, obj]{Web Utilities};
  \node (data2) [above=0.5cm of webutils]{Data};
  \node (api) [above=0.5cm of data2, obj]{API};

  \draw[->] (action) -- (dispatcher) -- (stores);
  \draw (stores) -- (inform);
  \draw[->] (inform) -- (views);
  \draw (views) -- (generate);
  \draw[->] (generate) -- (action);
  \draw[->] (webutils) -- (data) -- (action);
  \draw[->] (action) -- (data) -- (webutils);
  \draw[->] (api) -- (data2) -- (webutils);
  \draw[->] (webutils) -- (data2) -- (api);
\end{scope}
\end{tikzpicture}

\end{frame}